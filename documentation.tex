\documentclass{laspex}
%\documentclass[utan-onepages,without-entrances,utan-instruktioner]{laspex}
%\documentclass[book]{laspex}


\usepackage{blindtext}
\usepackage{todonotes}

\newcommand{\spextitel}{laspex-klassen}
\newcommand{\ellertitel}{typ}
%\newcommand{\korttitel}{M\&E}

\newcommand{\rom}{\replik{Romeo}}
\newcommand{\jul}{\replik{Juliette}}


\begin{document}
\begin{framsida}
\Huge \LaSPeX{}
\end{framsida}

\begin{karaktarsida}
\paragraph{Denna \LaTeX-klass} gör det enkelt att typsätta manus med sånger etc.
\end{karaktarsida}

\akt{De mest ofta använda kommandona}

\scenanvisning{En akt består av scenen, vilka i tur består av instruktioner och repliker, etc. Denna akt beskriver dessa kommandon.}

\scen{Akter och scener}

\jul    För att påbörja en ny akt, använd\\ \verb!\akt{}!.
        Om din akt har ett namn (t.ex., ``Akter och scener'') kan du i stället skriva\\ \verb!\akt{Akter och scener}!.
        
\forts  Och för att börja en ny scen, använd\\ \verb!\scen{}! eller \verb!\scen{Namn på scen}!.
        
\subscen{Delscener}
\jul    För att dela upp en scen i flera delscener, använd
        \verb!\subscen{}!, t.ex. \verb!\subscen{Delscener}!
        Detta kommer inte inkremera scenräknaren, men nya delscensräknaren.

\subscen*{Delscen och inkremera scenräknaren.}
\jul    Inkrementera scenräknaren en gång och nollställ subscenräknaren med stjärn\-varianten av kommandot\\ \verb!\subscen*{}!
        Då en scen ska börja med en subscen är detta rätt kommando.

\subscen{Delscen -- alternativt argument}{X}
\jul    För att manuellt ändra vilket tecken som används för delscenindelningen, använd\\ \verb!\subscen{Namn på scen}{X}!
        Detta stoppar inte inkrementering av delscensräknaren. Nästa delscen skulle heta 1.2c.

\scen{Repliker och instruktioner}

\subscen{Scenanvisningar}

\scenanvisning{Med kommandot} \verb!\scenanvisning{}! \scenanvisningforts{kan du skriva instruktioner för skådis.}

\scenanvisning{För att fortsätta på samma rad med en scenanvisning, använd}\\ \verb!\scenanvisningforts{}!\scenanvisningforts{.}

\jul    Detta är ett exempel! \scenanvisningforts{Och det ser ut såhär.}

\subscen{Repliker}
\jul    Hejhej! Och med \verb!\replik{}! skrivs en rolls namn ut i vänstermarginalen följt av ett kolon. T.ex., \verb!\replik{Romeo}!.
\rom    Och om du vill effektivisera ditt skrivande kan du i preamble använda\\ \verb!\newcommand{\rom}{\replik{Romeo}}!
        vilket gör det möjligt att bara skriva\\ \verb!\rom Hejhej!,
        i stället för\\ \verb!\replik{Romeo} Hejhej!.
        
\forts  Skapa ett ``tom'' replik med\\ \verb!\forts!.
        Snyggt att använda efter t.ex. scenanvisningar eftersom lite vertikal utrymme skapas ovanför. Också bra för att dela upp repliker i stycken.

\subscen{Hur, till, och andra kommentarer}


        
\scen{Musiknummer}

\jul    Om någon ska sjunga, använd\\ \verb!\musikbox{}{}{}{}{}!,
        t.ex.\\ \verb!\musikbox{Titel på numret}{Vilka som medverkar}%!\\ \verb!{Beskrivning av vad som pågår}{Originaltitel}%! \\ \verb!{Originalkompositör}!

\musikbox{Titel på numret}{Vilka som medverkar}{Beskrivning av vad som pågår}{Originaltitel}{Originalkompositör}

\jul    Du kan lägga till en undertitel genom att lägga till en på slutet med  t.ex.\\ \verb!\musikbox{Titel på numret}{Vilka som medverkar}%!\\ \verb!{Beskrivning av vad som pågår}{Originaltitel}%!\\ \verb!{Originalkompositör}{med en undertitel}!

\musikbox{Titel på numret}{Vilka som medverkar}{Beskrivning av vad som pågår}{Originaltitel}{Originalkompositör}{med en undertitel}

\subscen{TO DO Kommentar}
\rom    Ska komma ihåg att skapa en variant som inte skriver ut (/) om argument 4 och 5 är tomma, såsom i
    \verb!\musikbox{Titel på numret}{Vilka som medverkar}%!
    \verb!{Beskrivning av vad som pågår}{}%!
    \verb!{}!
    vilket nu ser ut som

\musikbox{Titel på numret}{Vilka som medverkar}{Beskrivning av vad som pågår}{}{}

\scen{Dekor, teknik, och andra instruktioner}

\scen{Rörelser på scen, för inspicient}

\jul    Om inspicienten vill enkelt se när roller går av och på scen kan de använda kommandona\\ \verb!\ingang{}! och\\ \verb!\utgang{}!
        T.ex., säg att Romeo kommer på att hen ska gå av scenen. Då kan det skrivas såhär i manus:

\utgang{Romeo går ut.} \scenanvisningforts{ Romeo får ett infall att rusa ut.}
\scenanvisning{Visar sig att han gjorde fel och vänder in igen.\ingang{Romeo går in.}}

\subscen{TODO Kommentar till rörelser på scenen}
\rom    Varför påbörjas ny rad när ingång används? Så ska det väl inte vara? Lägg till \nolinebreak som i scenanvisningforts

\scen{Alternativ för klassen}

\akt{Bok-alternativet}


\end{document}
