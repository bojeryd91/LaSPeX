\documentclass{laspex}
%\documentclass[utan-onepages,without-entrances,utan-instruktioner]{laspex}
%\documentclass[book]{laspex}


\usepackage{blindtext}
\usepackage{todonotes}

\newcommand{\spextitel}{Maria Stuart och Elisabet I}
\newcommand{\ellertitel}{slottsdamer i skottdrama}
\newcommand{\korttitel}{M\&E}

\newcommand{\ms}{\replik{Maria}}
\newcommand{\eli}{\replik{Elisabet}}
\newcommand{\se}{\replik{Seton}}
\newcommand{\dud}{\replik{Dudley}}
\newcommand{\bac}{\replik{Bacon}}
\newcommand{\har}{\replik{Härolden}}
\newcommand{\fil}{\replik{Filip}}
\newcommand{\liv}{\replik{Livingstone}}
\newcommand{\fle}{\replik{Fleming}}
\newcommand{\bea}{\replik{Beaton}}


\begin{document}
\begin{framsida}
Här kan det vara en bild, eller vad du vill!
\end{framsida}

\begin{karaktarsida}
\paragraph{Roller:}
\paragraph{Maria Stuart} är nykommen till det engelska hovet. Hon är ensam och vill få nya vänner, och kan tänka sig att anpassa sig en hel del för att passa in. Hon ser fram emot att lära känna sin kusin Elisabet 

\paragraph{Elisabet I} är drottning i England, och försöker styra riket som en stark sådan. Hennes vänner är Bacon och Dudley. Hon har det jobbigt med ett spöke till far, Henrik VIII, som vill styra från andra sidan. Hon är nervös inför att Maria ska komma.

\paragraph{Mary Seton} är coolast på hovet och ledaren för ett gäng av marysar. Hon och Maria känner varandra från barndomen, och Seton tycker att Elisabet är en väldigt ocool drottning. 

\paragraph{Robert Dudley} framstår som trevlig och charmig, men är i själva verket främst egenkär. Han har ett stort bekräftelsebehov och stöter gärna på damer han möter. Om (när) han blir avvisad blir han istället bitter och avundsjuk.

%är trevlig, charmig och populär, men har också ett stort bekräftelsebehov och söker ``kärlek'' på det där sätt som ungdomar gör. Tidigt i första akten stöter han på Maria, och senare i akt 2 på Elisabet, som båda avvisar honom. Detta tar fram hans bittra och svartsjuka sida.

\paragraph{Francis Bacon} är nördkompis, bundsförvant och bollplank för Elisabet. Han är Elisabets närmste vän och tycker om att filosofera, något han märker att även Filip gör.

\paragraph{Filip II} har skickats till det engelska hovet med ett uppdrag från sin farmor: fria till drottning Elisabet. Detta är inte något han själv vill göra, men om farmor tjatar... Han tycker om att filosofera och är en mystisk bad boy som av misstag stöter sig med Dudley när han faller för Maria.

%är av mystisk anledning vid Engelska hovet. Vill han köpa virke till sin spanska armada? Blivit ditskickad mot sin vilja? Svår och ``djup'' kille som Maria blir kär i. 

\paragraph{Marysarna} heter Mary Livingstone, Mary Fleming och Mary Beaton. De är tre hovdamer som hänger efter Seton. Livingstone är inte supersmart.

\paragraph{Henrik VIII.} Spöke och far åt Elisabet. Tycker att halshuggning löser de flesta problem som kan uppstå.

\paragraph{Härolden.} Medeltida sociala medium.

\paragraph{Vakt.} Vakt i fängelset i akt 3.

Sen lägger jag till text för att se vad som händer om karaktärssidan blir mer än en sida lång. Förhoppningsvis händer inget konstigt, utan sidnumret ökar från i till ii och formatteringen håller.

Så vad händer? Hände något? Dags att kompilera och se!

Visade sig vara för kort \dots{} så jag fortsätter improvisera och nu kanske jag ska googla hur jag slänger in blindtext

\blindtext[4]

\end{karaktarsida}

%Bra kommandon: \hur{}, \till{}, {\it }, \dekor{},\kommentar \dots{}{} \nytt{}
% \musikbox{Låttitel}{Karaktärer som sjunger}{Originaltitel}{Originalkompositör}{Vad den handlar om}
% \replik{nykaraktär} kan användas till en karaktär som bara används någon gång.

\akt{Den första akten}

\scenanvisning{Ridån går upp och ett slott eller något står i bakgrunden.}

%%%%%%%%%%%%%%%%%%%%%%%%%%%%%%%%%%%%%%%%%%%%%%%%%%%%%%%%%%%%%%%%%%%%%%%%%%%%%%%%%%%%%%%%%
\scen{The Trial}
\ingang{Elisabet in vänster framför ridå.}
\scenanvisning{I slutet av ouvertyren blir musiken allvarligare, Maria och Vakten kommer in. Maria ställer sig på mitten av scenen.}
%\teknik{Skapar denna extra whitespace i manus?}
\margintext{Dekor}{Skapar denna extra whitespace i manus?}
%\marginpar{\texttt{denna då?}}

\eli Efter allt vi bevittnat kan vi, Elisabet av England, nu besluta,
		att ni, Maria Stuart av Skottland, döms till döden.
		
\subscen{Ny scen med subscener}
\ms     Av alla tänkbara öden, \kommentar{Med dramatisk gest.}
\utgang{Elisabet och vakten ut vänster}
\forts  var det inte så här jag ville att det skulle sluta.
		
		\till{publiken} Ni undrar säkert hur vi hamnade i den här s(m)örjan,
när allt såg så bra ut till en början.
		Alla historier har två sidor att berätta,
		nu ska jag framföra min, den rätta!

\subscen{Nästa subscen med godtyckligt namn}{X}

\scenanvisning{Ridå upp.}
\ingang{Skådis, kör och dans på scen bakom ridån}

\musikbox{Marias dröm}{Alla}{Storslaget introduktionsnummer.}{We're all in this together}{High School musical}{om hur livet på hovet kommer att bli}

\utgang{Alla av scen, Bacon och Dudley först, Maria sist}

%%%%%%%%%%%%%%%%%%%%%%%%%%%%%%%%%%%%%%%%%%%%%%%%%%%%%%%%%%%%%%%%%%%%%%%%%%%%%%%%%%%%%%%%%
\scen{Maria träffar Bacon och Dudley}
\scenanvisning{Maria går av scenen, och ut från slottsporten kommer Bacon och Dudley. Bacon håller i en vinflaska, Dudley i en whiskeyflaska.}\dekor{flaskor}
\ingang{Bacon och Dudley in, genom porten.}
\dud	\dots{} och därför är jag välkomstkommité,
		och du Bacon får stå här brevé.
\subscen{En subscen}

\bac \babord{Så jag fixar snacksen,
och du sköter allt snack sen?
Sen fortsätter han säga något väldigt omständigt och\\ typiskt hans karaktär för att raden ska bli väldigt lång.
Han kanske utbrister i en sång!}

\dud \mittbord{\kommentar{Ignorerar Bacon.} En skotsk drottning borde välkomnas med en \emph{skott-tår.} \kommentar{Håller upp en flaska whisky.}
Sen säger han något mer
som du ser.
Sluta prata, jag ber
dig, annars ropar jag på fler
tills detta stycke trillar ner
på nästa sida
och då får du lida.
Detta genererade overfull box
vilket inte var min tanke.
Får gå på LaTeX-detox
och sen tillbe Ananke.}

\bac \styrbord{\hur{Besserwissrigt} Dudley, hon flyttade faktiskt till Frankrike som ung,
		och giftes bort till deras dåvarande kung.
Så kanske en flaska bordeaux,
från -32? 
\hur{Resonerande} Vilket, när den gregorianska kalendern införs, \textbf{kommer att ha varit} ett skottår. 
\kommentar{Dudley har vid det här laget tröttnat på Bacons utlägg.}}

\dud \kommentar{Suckar djupt:} Spring och hämta nån karl 
som kan spela en välkomstfanfar.
\bac	\kommentar{Pekar på orkestern} Men Dudley, vi har ju redan en adekvat orkester här! 
\dud	Ha! Den där?
		Det måste du väl begripa,
att det inte duger med orkester som saknar både takt och ton,
hyfs och fason,
franskt horn, och framförallt en säckpipa!
\utgang{Bacon ut höger bak.}
 
%%%%%%%%%%%%%%%%%%%%%%%%%%%%%%%%%%%%%%%%%%%%%%%%%%%%%%%%%%%%%%%%%%%%%%%%%%%%%%%%%%%%%%%%%
\akt{} 

\ingang{Maria kommer in vänster bak}\scenanvisningforts{förklädd, och Dudley förstår inte att hon är Drottning Maria.}

\scen{Första scenen i andra akten}

\ms		Ursäkta, hej!
\dud	Jag är inte intresserad. \kommentar{Vänder sig bort.}
\ms 		\kommentar{Söker kontakt.} Är det här slottet Fotheringhay?
\dud	Ja, men du är ju feladresserad. 
Köksingången är där, bortom slänten.
Och jag väntar faktiskt på den skotska regenten.  \kommentar{Vänder sig bort igen, gör “talk to the hand”.} 
\ms		Då är min resa finito.
		Jag kan äntligen sluta resa inkognito. 
\scenanvisning{Maria tar av sig täckmanteln och räcker fram en hand åt Dudley.}
\forts		Maria Stuart, drottning av skottarnas land.
\scenanvisning{Dudley inser sitt misstag.}
\dud	Oj … öhh \dots jag menar asså att Ert entourage, 
		kan gå till köksingången med Ert bagage.
		Lord Robert Dudley, earl av Leicester.		\kommentar{Uttala ``Läster'' (så det rimmar på ``gäster'')}
		Vad trevligt med lite rojala gäster
\ms		Merci, men jag reste på egen hand.
\dud	Åhå! \kommentar{Nu blev det plötsligt väldigt intressant!}
Så \dots{}
vad hände med er kung, er fransman?
\ms		Han försvann.
Det växte en stor böld i hans ena öra
och på läkarnas råd kunde han inte höra. 
\dud	Det förklarar \emph{änke}-biljetten, ers höghet   
men varför reste ni så diskret? 
\scenanvisning{Dudley är inte jätteintresserad av svaret utan kan under dessa rader fixa till håret, använda munspray o.s.v. utan att Maria ser detta.}
\ms		På begravningen gjorde jag en politisk groda: 
		när jag sa att paddor var minst lika goda,
		Detta skapade en nationell kris. 
		Så för att behålla mitt eget skinn
		och huvudet över halsen min
		sprang jag hals över huvud från Paris.
\dud	\hur{Förföriskt} Det låter som att det var långt att ila? 
\ms		Ja, men jag ville undvika att bila. 
\dud	Och att dra till Skottland var inte möjligt? 
\ms		Mais oui, det kanske låter lite löjligt,			\kommentar{Uttalas: ``Mä wii''}
men mitt franska sätt verkar inte det skotska folket tåla
så de har satt skottpengar på mig.
Silverpenningar till beloppet två.
\dud	Åh nej!
		\hur{Charmigt} Jag tycker ni är mycket mer värd än så. 
\ms		Jo, men vi skottar är rätt snåla. 
\dud	Sån är inte en engelsk gentleman.
Är du villig, 
kan jag bjuda på en middag som är grann,
och allt annat än billig.
Kanske fish and chips till en het fri tös? 
\ingang{Bacon in höger bak} \scenanvisningforts{(med ett instrument som inte är en säckpipa? En säck och en pipa?). Inser att Maria är Maria.}
\ms		Ni är väldigt generös, 
men tyvärr åt jag en lokal petit dejeuner,		\kommentar{Uttalas: ``de-sjö-né''.}
med bröd, och korv, och te, \kommentar{Andning.} 
och vita bönor, och stekt svamp, och pudding, och ägg åå\dots{} 	\kommentar{Omstart: \dots{} och spam, \dots{} och spam \dots{} }
\bac	\kommentar{Sträcker fram hand.}  Bacon, lordkansler, justitiekansler, och förste filosofie förstå\-sig\-påare-kansler, samt innehavare av fjärde ordens strumpeband.
\ms		Wow, att du hinner med allt det där.
\dud	Nej, det gör han inte, det går sisådär,
så vi brukar kalla Bacon för Lord \emph{Kass}-ler ibland.
\bac Har Ni några allergier,
eller fobier?
\ms		Ja, jag har båda två:
		Bordeaux, Bordeaux.
        
 %%%%%%%%%%%%%%%%%%%%%%%%%%%%%%%%%%%%%%%%%%%%%%%%%%%%%%%%%%%%%%%%%%%%%%%%%%%%%%%%%%%%%%%%%
\scen{Maria lär känna Mary${}^4$}

\ingang{Seton in vänster bak.}
\se \hur{Superdupertaggat} Aaaaaaaah! Maria!
\ms Mary Seton, är det du?
\se Ja, ja, ja – ja gånger sju!
\ms Senast vi sågs var innan den franske kungen fria’.
\se Vi har sååååå mycket att ta igen.
\kommentar{Seton upptäcker att Bacon och Dudley är där.}
\se Åh,
Lord Dudley, Lord Kass\dots{} Lordkansler Bacon: Hallå.
\dud Lady Seton \dots{} 
\bac Lady Seton. Vårt hemliga uppdrag är slutfört så \dots
\dud \dots{} vi borde nog gå.
\utgang{Bacon och Dudley ut genom porten.}
\se Jag har en väldigt lång, intressant och rolig historia jag måste berätta, men,
		först måste jag skriva på Twitter att du har kommit hit! \kommentar{Börjar skriva på en lapp.} 
\ms Vad är en Twitter?
\se Va? Det är ju det bästa som hänt sen min brevduva drog sitt sista \dots{} kvitter! 
		Du måste skapa dig en profil så att du kan tacka ja till min vänskapsinvit. \hur{Klappar i händerna}
        
\ingang{Härolden in vänster bak.}

\forts	Med sociala medier kan du utan att röra ett finger nå ut med ett budskap till vem som helst.
\har \hur{Till publiken} Hashtag: Facebookfrälst.
\ms Så härolden förkunnar det du skriver till dina vänner?
\se Ja, och även till folk jag inte känner.
		Det är helt genialt!
Man kan dela skisser på det jag ska äta, overifierad fakta, söta katter, med mera.
\ms Men hur är sociala medier egentligen socialt\dots 
\se Nej, nej, låt mig demonstrera:
\scenanvisning{Seton räcker över en pappersremsa till Härolden.}
\har ``Kungligt besök. Ni kommer inte tro att detta är sant!
\#var-här-eller-var-en-fyrkant''

\utgang{Härolden ut vänster bak.}

\se Här kommer mitt häng.
		Jag och mina lojala följare är hovets populäraste gäng.
Detta är Mary Beaton,
\ingang{Beaton in vänster bak} 
\forts Mary Fleming,
\ingang{Fleming in vänster bak}
\forts och Mary \dots
\ingang{Livingstone in vänster bak}
\ms \dots{} Livingstone, förmodar jag?
		Ça va?
\bredReplik{Livingstone, Seton, Fleming, mamma, pappa, barn, mormor, morfar, farmor, farfar \& Beaton}
    \dots{} Va' sa?
\se Ursäkta hennes franska.
\fle Det är ju svårare att förstå än danska.
\bea Asså jag tror hon sa: God dag, 
		\hur{Väldigt tveksamt} eller: min svävare är full av \dots{} säl?
\se Maria Stuart här är Elisabets skotska kusin.
\ms Känner ni drottningen väl?
\se \dots{} eeeh \dots{} Jo, ja, hon är \dots
\bea söt \dots{} 
\liv \dots{} som en sockerfri pralin,
\fle och frän som \dots
\bea \dots{} klorin.
\liv och torr som \dots
\fle \dots{} ett dessertvin!
\ms Låter ju som en smakfull tjej.
\se Men hon kommer nog bli jätteglad över att se dig.
\ms Borde jag tänka på nåt speciellt när jag träffar Elisabet?

\scenanvisning{Mary B, Mary F och Mary L börjar viska hetsigt till varandra.}

\se Nej, nej, du är bra som du är.
\bea Men det skulle inte skada att ändra lite på det där,
\fle \dots byta ut den här,
\liv och göra om hela din karaktär.
\se Om det inte är till för mycket besvär
		är det dags att slopa ditt fransk-skotska sätt.


\musikbox{Makeover av Maria}{Alla som har ett namn som börjar på M}{Sång om hur Maria ska göras om för att passa in i engelska hovet. Hon ska bli mer engelsk. }{Free your mind+Bang Bang}{En Vogue+Jessie J}

\bea Tänk bara på precis allt vi just lärt dig.
\liv Då kommer allt lösa sig.
\fle Nu drar vi och stajlar om en annan tjej. 

\utgang{Beaton, Fleming, Livingstone ut vänster bak.}

%%%%%%%%%%%%%%%%%%%%%%%%%%%%%%%%%%%%%%%%%%%%%%%%%%%%%%%%%%%%%%%%%%%%%%%%%%%%%%%%%%%%%%%%%
\scen{Maria träffar Elisabet}
\scenanvisning{Mary Seton håller sig undan och Maria går fram och knackar på porten. Härolden öppnar och tittar/kliver ut.}
\ingang{Härolden in genom porten.}

\har	Drottning Elisabet av England och Irland meddelar dem som vill göra entré:
		``BRB''
\scenanvisning{Maria ser frågande ut.}
\ms 	BRB?

\se Det betyder: “Brittiska Regenten Borta”.
		Härolden gillar å förkorta. 
\har Engelska hovet använder sig av kakor för att förbättra Er upplevelse av vårt slott.
\kommentar{Sträcker fram en kaka.}
\ms Åh, vad gott!
		\till{Härolden} Jag accepterar, får jag komma in då?

\scenanvisning{Härolden tittar in i slottet, får en lapp och läser från den.}

\har Nej. Du får en kölapp till plats fyrtiotvå.
\kommentar{håller fram en kölappsmatare.}
		GTG, måste gå.
\utgang{Härolden ut genom porten} \scenanvisningforts{och stänger porten. Seton går fram.}

\se Du måste vara mer enträgen. \kommentar{Börjar banka på porten.}
		Hallå?
Vart tog du vägen?
\scenanvisning{Dörren börjar öppnas, och Seton springer och gömmer sig. Elisabet tittar ut genom porten. På sitt huvud har hon ett får, som på sitt huvud har en basker med röd fluffboll (se Balmoral). Kanske dekorerad med skotska och franska flaggor? \dekor{Får med balmoral!}}
\ingang{Elisabet in genom porten.}

\ms Hej kära kusin,
long time no see\dots{}n.
\eli Hej \dots{}

\scenanvisning{Det blir tyst en stund. Seton gör tummen upp etc. till Maria.}

\ms \kommentar{Pinsamt försök att komma igång:} Heeeeeeej, på deeeeeeej!

\scenanvisning{Maria lutar sig framåt för att kindpussa Elisabet.}

\eli Nej! \kommentar{Föser Maria tillbaka.}

\scenanvisning{Seton gestikulerar (borstar utmed en klänning) att Maria ska kommentera Elisabets klänning.}

\ms Jag \dots{} tror att du typ har lite damm på din kjol \kommentar{Kort paus.} 
		som för övrigt är SÅ cool.
\eli Okej
\kommentar{Seton gestikulerar vidare, pekar på huvudet.}
\ms		Öh \dots{} en fråga, om jag får:
		\hur{Positivt} Vad har du gjort med ditt hår?
\eli \hur{Offensivt} Vadå’rå? \kommentar{Tar snabbt av sig fåret och kastar in det i slottet.}
\ms Nej, asså jag tycker inte heller om får å \dots{} 
		andra skotska och franska influenser.
Usch för allting utanför Englands gränser\dots 
\eli Jaså, det säger du?
\ms Ja, det är så sjukt skönt att slippa allt det där, 
och äntligen vara här! 
\eli \hur{Kallt} Jo, verkligen, bra att du kom \dots{} precis nu.
\ms Asså, jag har på känn, 
att du kommer bli min bästa vän.		
\eli Ja men \dots{}
		vi ses sen. \kommentar{Stänger dörren.}
\utgang{Elisabet ut genom porten.}

\ms Det där,
det hade ju inte kunnat gå mycket värre.
\se Jo, du har i alla fall huvudet kvar,
		medan andra har betydligt färre. 
		Elisabet brås ju lite på sin far. 
\ms Men inte gjorde väl morbror Henrik någonsin en fruga förnär? 

%%%%%%%%%%%%%%%%%%%%%%%%%%%%%%%%%%%%%%%%%%%%%%%%%%%%%%%%%%%%%%%%%%%%%%%%%%%%%%%%%%%%%%%%%
\scen{Filips entré}

\ingang{Härolden kommer in genom porten.}
\har	Filip II, Spaniens monark
		har precis checkat in i slottets park.
        \ingang{Beaton, Fleming, Livingstone in vänster bak.}
       \ingang{Kör in genom porten}
		\forts Vill ni checka in hans entré
		bör ni också göra de’.
\ingang{Dans och Filip in från höger och vänster bak under numrets intro.}

\musikbox{Filips entré}{Filip + folk}{Filip kommer in och presenteras}{Manboy}{Eric Saade}
\utgang{De flesta dansarna ut höger och vänster bak.}
\fil	Om ett träd faller i skogen, men ingen ser skogen för alla träden,
		hur många fler träd måste falla innan man kan se den?
		Och var gör man sedan med all veden?
		Bygger en armada grande och åker ut på den spanska räden?
\kommentar{Omstart: Den spanska räden red en annan räd.}

\scenanvisning{Dudley står en bit bort.}

\dud	Bah! Han skulle bara våga!
\replik{Statist}	Det var säkert inte mer än en retorisk fråga.

\liv		Åh, han verkar både klyftig, mogen, och söt,
\fle	som en apelsin,
\bea	mandarin,
\liv	eller kokosnöt!

\fil Att det finns liv efter döden står skrivet,
men finns det död efter livet?
\replik{Alla statister}	Mm \dots{}

\scenanvisning{Dudley lyssnar bakom och blir svartsjuk.}

\liv		Vilken läckerbit!
\fle	Smäller högre än dynamit
\bea	Bad boy som en bandit,
\liv	och coolare än en inuit.
\fle	Mitt huvud snurrar som en satellit
\ms		Satellit?
\liv	åh \dots{}
\bea	\dots{} åh \dots{}
\fle	\dots{} åh \dots{}
\ms 	Varför tror ni han har kommit hit?
\se 	Hans motiv är uppenbart:
\bea 	Han är en spansk spion!
\liv 	Nä, han förbereder en invasion!
\fle 	Han vill få oss att byta religion,
		med sin spanska inkvisition.
\se 	Nej, jag tror att han tänker fria till Drottning Elisabet och ta över hennes tron.
\fle 	Ja, såklart!
\bea 	Smart!
\liv 	Element\emph{art}!

\ms \hur{Besviket} Jahapp \dots{} Där rök chansen
		för den romansen.
\ingang{Bacon in genom porten.}        
\scenanvisning{Dudley står i närheten av Maria.}

\dud 	Varför kan regenter inte bara stanna hemma och styra sitt eget land?
\ms 	\hur{För sig själv} Det undrar jag också ibland.
\utgang{Härolden ut genom porten.}
\dud 	Han passar inte in här med sina konstiga kläder och språk.
\ms 	\hur{För sig själv} Främmande kulturer kan vara svåra att förstå.
\dud	Såna här statsbesökare leder alltid till bråk.
\ms		\hur{Syftar på Filip II} Vi har så mycket gemensamt vi två!
\dud Han tror att han är värsta snubben.
Man ba, sitt ned i båten lilla gubben!
\utgang{Dudley, körare och dansare ut genom porten.}

\scenanvisning{Maria tittar bort mot Filip igen.}

\ms 	Jag tycker det är trevligt att någon här vågar vara den hen är.
\se 	Jag tror att någon är lite kär!
\ms 	Det är jag inte alls,
		din lögnhals!
\se 	Jag ser bara vad jag ser.
\ms		Att stöta mig ännu mer med Elisabet, 
		å stöta på hennes kavaljer, 
		vore dessutom inte rätt.
\se 	Men om han skulle uppvakta dej,
		behöver du väl inte akta dej?

\scenanvisning{Filip tittar på Maria och försöker fånga hennes blick.}

\forts	OMG, Maria, titta inte nu,
		men Fräsiga Filip spanar in dig klockan tre.
\ingang{Härolden in genom porten.}        
\scenanvisning{Alla utom Livingstone vänder sig om åt rätt håll. Livingstone kollar omkring sig förvirrat och håller upp ett timglas.}

\liv	När då sa du?
\bea	GeäMTe.

\scenanvisning{Livingstone vänder sig rätt. Filip blir skrämd och kollar bort och ner i marken. Bacon går fram till Filip och börjar prata med honom.}

\scenanvisning{Härolden går fram till Maria.}

\har 	Inbjudan till Välkomstfest,
		O-S-A senast nu i da’:
		Nej, intresserad, eller ja?

\utgang{Beaton, Fleming Livingstone ut genom porten.}

\se 	Klart du ska gå!
\utgang{Härolden ut höger bak.}
\ms 	Varför då?
\se 	Du är hedersgäst,
		och jag är säker på att Filip ska attenda.

\utgang{Maria ut genom porten.}

\forts	\hur{Lömskt för sig själv} En bal på slottet kan ju få vad som helst att hända.

%%%%%%%%%%%%%%%%%%%%%%%%%%%%%%%%%%%%%%%%%%%%%%%%%%%%%%%%%%%%%%%%%%%%%%%%%%%%%%%%%%%%%%%%%%%
\scen{}
\ingang{Bacon och Filip in vänster bak.}

\fil	\dots{}  så jag söker svaret på livets stora frågor.
\bac	Stora frågor gör mig eld och lågor!
		Med rätt resonemang kan man komma fram till det mesta.
\se		Till exempel att det vore kul att gå in och festa.
\fil Jag tar hellre en siesta.
\se Hur kan du veta det utan att först testa?
\bac Nej, experiment är inte vägen till sanningen
Man resonerar sig fram till rätt svar!
\se Och det gör man väl bäst vid en bar?
\bac \till{Filip} Om man får fel svar, börjar man om igen.
\fil Bacon, så tru,
jag tänker precis som du!
\replik{Filip \& Bacon} Alltså är vi!
\bac Genom tankens kraft kan en förstå allt och ingenting!
\fil Värsta genombrottet!
\se Och jag tänker mig att ni,
ska gå på balen i kväll!
		Så då måste det vara så, om man är rationell.
\till{Filip} Du kanske till och med hittar en fling!
\fil Men, vad e’ väl en bal på slottet?
Den kan vara långtråkig,
\bac dötrist\dots 
\fil \hspace{10pt} \dots bråkig \dots 
\replik{Filip \& Bacon} \dots och alldeles, alldeles \dots{} 
\se \dots{} öppen bar,
		vet jag att de har!
\fil Men, vad e’ väl en bar på slottet?
\bac Ja, i normala fall,
		hundra kiloPascal
		-- som jag har förstått det.
\fil Men Bacon, vad e’ det för mening att besöka en bar?
\se Vad är det för mening med att \emph{inte} göra det?
\bac Det är det ingen som vet \dots{}

\se		Festen kommer börja med lite mingel.
		Då kan du ju träffa Maria, som på tal om ingenting är singel.

\fil	\hur{Mer peppat} Då kan vi lika gärna gå
		båda två.

\utgang{Seton, Bacon, Filip ut genom porten.}

\scenanvisning{\textbf{Det här är en beskrivning av saker som händer på scenen i mellanspelet och matchar med vad som händer på festen i akt 2.}} 
\scenanvisning{Mellan 1.6 och 1.7: ljus som blinkar genom fönstren till slottet. Orkester-fest-musik.} 		
\ingang{Vakten och bröderna Bacon in från porten.}

\begin{onepage}
Detta är ett exempel på en vänstersida i en spexbok. Med the environment \texttt{onepage} kan du lägga in text/material som ska flyta in på \textbf{en} egen sida i boken. Nedan jag har slängt in ett \texttt{$\backslash$vfill} samt random text för att illustrera att det trycks ned till botten av sidan.
\vfill
\blindtext[3]
\end{onepage}

%%%%%%%%%%%%%%%%%%%%%%%%%%%%%%%%%%%%%%%%%%%%%%%%%%%%%%%%%%%%%%%%%%%%%%%%%%%%%%%%%%%%%%%%%%%
\scen{}

\scenanvisning{Festen pågår en stund. Härolden går över scenen med en skylt där det står "fest forward". Filip kommer ut och börjar läsa en bok. Maria kommer ut och verkar berusat betuttad. \dekor{Skylt med texten "Fest forward"}}
\ingang{Filip in genom porten.}
\ingang{Maria in genom porten.}

\ms		\kommentar{Upptäcker Filip}
Är en så komplex person som han på riktigt, eller rent imaginär?
\kommentar{Maria går fram till Filip.}
Vad gör en drömprins som du utanför ett sagoslott som det här?
\fil Jag läser Martin Luthers lilla katekes.
\ms Visst är han en fantastisk berättare.
\fil Jag förstår mig faktiskt inte på hans huvudtes.
\ms \dots{} fantastiskt oklar är han, denne luddige kättare!
\fil Ifall han inte skrivit på tyska hade det varit lättare.
		Men med resten av teserna slår han huvudet på en spik …
\ms Ja!
\fil \dots{} kan man tycka om man inte är katolik.
\ms \dots{} den där är verkligen inte bra. \kommentar{Syftar på boken.}
Bara vidskepelse med myter i.
\fil Vilken diskutabel filosofi.
\ms Ja, i min mening hade Europa varit en konfliktfri union 
under den katolska tron
		om det inte varit för den där störige Luther
Det är i alla fall ingen som blir putt här
Ty England lyder icke under Påven
\fil Vad skönt att du inte är en sån där Mary, Maria
Jag har blivit anfallen av en hel armada av Marys där inne på festen
Där varenda en är den andras masskopia 
Precis som på de andra europeiska hoven...
\ms Förresten, 
på tal om ingenting,
ser jag ingen ring.
\fil Nej, det beror på att jag är lyckligt ogift
		Tyvärr har jag skickats hit med frieri som uppgift
		Personligen föredrar jag att fly framför att illa fäkta,	
Men min farmor, mi abuela, är väldigt envis.
Hon vill att jag ska hitta mig en brittisk drottning att äkta
\ms \kommentar{För sig själv} Oflyt, då ligger jag i lä …
		Vänta nu, brittisk drottning, det är ju jag mä.
\fil \kommentar{Insiktsfullt} Där har du rätt!
\ms \kommentar{Full, med försök till spansk brytning} Din farmor Mia-Bella är sååå vis!
\fil \kommentar{Försiktigt} Då är ju du rent tekniskt ett alternativ till Elisabet.
\ms Ja och vi vill ju inte göra din farmor Salmonella besviken.	
\fil Jag kan inte hitta några luckor i den logiken.
		Trots din spanska är du inte dum.
\ms Nä precis, så\dots vill du hänga med och inta en Bloody Mary inne på mitt rum?
\fil Just nu verkar du vara rätt full av andra intag.
\ms Ahh tack, vad du är snäll!
		“Intagande” är det finaste någon har sagt om mig.
\fil Men jag ses gärna nån annan dag.
Till dess kommer mina tankar att ockuperas av dig
\ms Kan du avsätta tid för en drottning imorgon kväll?

%%%%%%%%%%%%%%%%%%%%%%%%%%%%%%%%%%%%%%%%%%%%%%%%%%%%%%%%%%%%%%%%%%%%%%%%%%%%%%%%%%%%%%%%%%%
\scen{}
\ingang{Elisabet in genom porten.}
\ingang{Dudley, Bacon, Seton, kören in genom porten.}

\scenanvisning{Elisabet slår upp portarna till slottet och marscherar ut. Folk följer efter, nyfikna. Vakten omringar Maria och Filip.}

\ms Åh, Elisabet, salut!
\eli Vi har allt sett vad Ni har i Er kikare
		din otacksamma svikare!
\ms Det är inte exakt som det ser ut. \kommentar{Står fortfarande nära Filip.}
\eli Håll trut.
\eli	Vi fördömer Er fördömda allians!
\fil Att vi skulle ha nånting på gång är inte sant, \hur{Till Maria} inte sant?
\ms \hur{För sig själv} Jag fick inte ens chans att fråga chans \dots{}
\eli Ni har i alla fall dansat Er sista dans.
\eli För förrädaren till vår mörkaste fängelsecell! \kommentar{Pekar -- med hela handen!}
	 Hennes dom avkunnas under morgondagen!
\bac Vår mörkaste fängelsecell är en \emph{aning} upptagen.
\eli \dots{} För förrädaren till den mörkaste fängelsecellen, som är tom!
		Redan ikväll,
får hon sin dom.


\musikbox{Maria har fängslats!}{Alla}{Maria ska fängslas! Vad kommer att hända med henne?  Har Elisabet överreagerat? Maria undrar hur hon hamnade här.}{I believe in a thing called love}{The Darkness}{A subtitle}

\end{document}
