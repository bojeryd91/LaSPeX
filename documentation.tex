\documentclass{laspex}
%\documentclass[utan-onepages,without-entrances,utan-instruktioner]{laspex}
%\documentclass[book]{laspex}


\usepackage{blindtext}
\usepackage{todonotes}

\newcommand{\spextitel}{The laspex class}
\newcommand{\ellertitel}{A non-existing funny pun}
%\newcommand{\korttitel}{M\&E}

\newcommand{\rom}{\replik{Romeo}}
\newcommand{\jul}{\replik{Juliette}}


\begin{document}
\begin{framsida}
\Huge \LaSPeX{}
\end{framsida}

\begin{karaktarsida}
\paragraph{Denna stilfil} gör det enkelt att typsätta manus med sånger etc.
\end{karaktarsida}

\akt{De mest ofta använda kommandona}

\scenanvisning{En akt består av scenen, vilka i tur består av instruktioner och repliker, etc. Denna akt beskriver dessa kommandon.}

\scen{Akter och scener}

\jul    För att påbörja en ny akt, använd
        \verb!\akt{}!.
        Om din akt har ett namn (t.ex., ``Akter och scener'') kan du i stället skriva
        \verb!\akt{Akter och scener}!.
        
\forts  Och för att börja en ny scen, använd
        \verb!\scen{}! eller \verb!\scen{Namn på scen}!.
        
\subscen{Delscener}
\jul    Ibland behöver en scen delas upp i flera delscener. Detta kan du göra med kommandot
        \verb!\subscen{}!, t.ex. \verb!\subscen{Delscener}!
        
\forts  Om du vill manuellt ändra vilket tecken som används för delscenindelningen, använd för kalla den X
        \verb!\subscen{Namn på scen}{X}!
        Se nedan. Detta stoppar inte inkrementering av delscensräknaren.

\subscen{Kommentar}{X}
\jul    Ännu inte bestämt om
        \verb!\subscen{}! eller \verb!\subscen*{}!
        ska stoppa inkrementering av scenräknaren.

\scen{Repliker och instruktioner}

\scenanvisning{Med kommandot} \verb!\scenanvisning{}! \scenanvisningforts{kan du skriva instruktioner för skådis.}

\jul    Hejhej! Och med \verb!\replik{}! skriv en rolls namn ut i vänstermarginalen följt av ett kolon. T.ex., \verb!\replik{Romeo}!.
\rom    Och om du vill effektivisera ditt skrivande kan du i preamble använda
        \verb!\newcommand{\rom}{\replik{Romeo}}!
        vilket gör det möjligt att bara skriva
        \verb!\rom Hejhej!,
        i stället för
        \verb!\replik{Romeo} Hejhej!.
        
\begin{itemize}
    \item forts,
    \item hur, till, kommentar
\end{itemize}

\replik{Vad händer?} Upptäckt bugg! Efter itemize blir det inte längre indrag vid radbryte inom en "rad", varför är det så?
        
\scen{Musiknummer}
\jul    Om någon ska sjunga, använd
        \verb!\musikbox{}{}{}{}{}!,
        t.ex.
        \verb!\musikbox{Titel på numret}{Vilka som medverkar}%!
        \verb!{Beskrivning av vad som pågår}{Originaltitel}%!
        \verb!{Originalkompositör}!

\musikbox{Titel på numret}{Vilka som medverkar}{Beskrivning av vad som pågår}{Originaltitel}{Originalkompositör}

\jul    Du kan lägga till en undertitel genom att lägga till en på slutet med  t.ex.
        \verb!\musikbox{Titel på numret}{Vilka som medverkar}%!
        \verb!{Beskrivning av vad som pågår}{Originaltitel}%!
        \verb!{Originalkompositör}{med en undertitel}!

\musikbox{Titel på numret}{Vilka som medverkar}{Beskrivning av vad som pågår}{Originaltitel}{Originalkompositör}{med en undertitel}

\subscen{Kommentar}
\rom    Ska komma ihåg att skapa en variant som inte skriver ut (/) om argument 4 och 5 är tomma, såsom i
    \verb!\musikbox{Titel på numret}{Vilka som medverkar}%!
    \verb!{Beskrivning av vad som pågår}{}%!
    \verb!{}!
    vilket nu ser ut som

\musikbox{Titel på numret}{Vilka som medverkar}{Beskrivning av vad som pågår}{}{}

\scen{Dekor, teknik, och andra instruktioner}

\scen{Rörelser på scen, för inspicient}

\jul    Om inspicienten vill enkelt se när roller går av och på scen kan de använda kommandona
    \verb!\ingang{}! och
    \verb!\utgang{}!
    T.ex., säg att Romeo kommer på att hen ska gå av scenen. Då kan det skrivas såhär i manus:

\scenanvisning{Romeo får ett infall att rusa ut.\utgang{Romeo går ut.}}

\subscen{Kommentar till rörelser på scenen}
\rom    Varför påbörjas ny rad när utgång används? Så ska det väl inte vara?

\scen{Alternativ att stänga av och på funktionalitet}

\akt{Bok-alternativet}


\end{document}
